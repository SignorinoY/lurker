\documentclass[cn]{elegantbook}
\usepackage[linesnumbered,lined,ruled]{algorithm2e}
\usepackage{listing,minted}
\setminted{
    frame=lines,
    framesep=2mm,
    baselinestretch=1.2,
    fontsize=\footnotesize,
    linenos
}
\title{}
\author{龚梓阳}
\date{\zhtoday}

\begin{document}

\frontmatter
\maketitle
\tableofcontents

\mainmatter
\chapter{引论}

\section{预备知识}

\subsection{范数}

\begin{definition}[范数]
    设 $X$ 是域 $\mathbb{K}$ (实数域或复数域)上的线性空间,函数 $\|\cdot\|:X\rightarrow\mathbb{R}$ 满足:
    \begin{enumerate}
        \item (正定性)$\forall x\in X, \|x\|\geq0$;$\|x\|=0\iff x=0$;
        \item (齐次性)$\forall x\in X, \alpha\in\mathbb{K}, \|\alpha x\|=|\alpha|\cdot\|x\|$;
        \item (次可加性)$\forall x,y\in X, \|x+y\|\leq\|x\|+\|y\|$。
    \end{enumerate}
    则称 $\|\cdot\|$ 是 $X$ 上的一个范数。
\end{definition}

\begin{example}[常见范数]
    常见范数有:
    \begin{enumerate}
        \item 空间 $\mathbb{R}$:$\forall x\in\mathbb{X}$
              \begin{equation}
                  \|x\|:=|x|
              \end{equation}
              即范数 $\|x\|$ 为 $x$ 的绝对值。
        \item 空间 $\mathbb{R}^{n}$:$\forall\mathbf{x}\in\mathbb{R}^{n}, \mathbf{x}=(x_{1},x_{2},\ldots,x_{n})^{\prime}$
              \begin{enumerate}
                  \item 欧几里得范数:
                        \begin{equation}
                            \|\mathbf{x}\|_{2}:=\sqrt{x_{1}^{2}+\cdots+x_{n}^{2}}
                        \end{equation}
                  \item  $l_{p}$ 范数 $(p\geq 1)$:
                        \begin{equation}
                            \|\mathbf{x}\|_{p}:=\left(\sum_{i=1}^{n}\left|x_{i}\right|^{p}\right)^{1/p}
                        \end{equation}
                        若 $p=1$,则
                        \begin{equation}
                            \|\mathbf{x}\|_{1}:=\sum_{i=1}^{n}|x_{i}|
                        \end{equation}
                        若 $p=\infty$,则
                        \begin{equation}
                            \|\mathbf{x}\|_{\infty}:=\max_{i=1,\ldots,n}|x_{i}|
                        \end{equation}
                  \item  $l_{0}$ 范数:
                        \begin{equation}
                            \|\mathbf{x}\|_{0}=|\{x_{i}:x_{i}\neq 0,i=1,\ldots,n\}|
                        \end{equation}
                        即向量 $\mathbf{x}$ 中分量 $x_{i}$ 不为零的个数。
              \end{enumerate}
    \end{enumerate}
\end{example}

\subsection{梯度向量与 Hessien 矩阵}

\begin{definition}[梯度向量]
    对于多变量函数 $f:\mathbb{R}^{n}\rightarrow\mathbb{R}$,即$f(\mathbf{x})=f\left(x_{1},x_{2},\cdots,x_{n}\right)$,如果 $f$ 在点 $\mathbf{x}_{0}$ 关于每一个变量 $x_{i}$ 都有偏导数 $\frac{\partial f}{\partial x_{i}}(\mathbf{x}_{0})$ 存在,则在点 $\mathbf{x}$ 上,这些偏导数定义了一个向量:
    \begin{equation}
        \nabla f(\mathbf{x}_{0})=\left(\frac{\partial f}{\partial x_{1}}(\mathbf{x}_{0}),\ldots,\frac{\partial f}{\partial x_{n}}(\mathbf{x}_{0})\right)^{\prime}
    \end{equation}
    该向量称为 $f$ 在点 $\mathbf{x}_{0}$ 的梯度向量。
\end{definition}

\begin{definition}[Hessian 矩阵]
    对于多变量函数 $f:\mathbb{R}^{n}\rightarrow\mathbb{R}$, 如果 $f$ 在点 $\mathbf{x}_{0}$ 的所有二阶偏导数都存在,那么函数 $f$ 在点 $\mathbf{x}_{0}$ 的 Hessian 矩阵为
    \begin{equation}
        \boldsymbol{H}(\mathbf{x}_{0})=\left[\begin{array}{cccc}
                \frac{\partial^{2} f}{\partial x_{1}^{2}}(\mathbf{x}_{0})            & \frac{\partial^{2} f}{\partial x_{1} \partial x_{2}}(\mathbf{x}_{0}) & \cdots & \frac{\partial^{2} f}{\partial x_{1} \partial x_{n}}(\mathbf{x}_{0}) \\
                \frac{\partial^{2} f}{\partial x_{2} \partial x_{1}}(\mathbf{x}_{0}) & \frac{\partial^{2} f}{\partial x_{2}^{2}}(\mathbf{x}_{0})            & \cdots & \frac{\partial^{2} f}{\partial x_{2} \partial x_{n}}(\mathbf{x}_{0}) \\
                                                                                     &                                                                      &        &                                                                      \\
                \vdots                                                               & \vdots                                                               & \ddots & \vdots                                                               \\
                \frac{\partial^{2} f}{\partial x_{n} \partial x_{1}}(\mathbf{x}_{0}) & \frac{\partial^{2} f}{\partial x_{n} \partial x_{2}}(\mathbf{x}_{0}) & \cdots & \frac{\partial^{2} f}{\partial x_{n}^{2}}(\mathbf{x}_{0})
            \end{array}\right]
    \end{equation}
    或使用下标记号表示为
    \begin{equation}
        \boldsymbol{H}_{i j}(\mathbf{x}_{0})=\frac{\partial^{2} f}{\partial x_{i} \partial x_{j}}(\mathbf{x}_{0})
    \end{equation}
\end{definition}

\subsection{凸集与凸函数}

\begin{definition}[凸集]
    设集合 $C\subset\mathbb{R}^{n}$。若对 $\forall\mathbf{x},\mathbf{y}\in C$,有
    \begin{equation}
        \theta \mathbf{x}+(1-\theta)\mathbf{y}\in C,\quad\theta\in[0,1]
    \end{equation}
    则称 $C$ 为凸集。
\end{definition}

\begin{definition}[凸函数]
    设集合 $C\subset\mathbb{R}^{n}$ 为非空凸集, 函数 $f:C\rightarrow\mathbb{R}$。若对 $\forall\mathbf{x},\mathbf{y}\in C$,有
    \begin{equation}
        f(\theta\mathbf{x}+(1-\theta)\mathbf{y})\leq\theta f(\mathbf{x})+(1-\theta)f(\mathbf{y}),\quad\theta\in[0,1]
    \end{equation}
    则称 $f$ 为 $C$ 上的凸函数。若上述不等式对 $\mathbf{x}\neq\mathbf{y}$ 严格成立,则称 $f$ 为 $C$ 上的严格凸函数。
\end{definition}

\begin{theorem}[凸函数的一阶判定条件]
    设集合 $C\subset\mathbb{R}^{n}$ 为非空开凸集,函数 $f:C\rightarrow\mathbb{R}$ 可微,则
    \begin{enumerate}
        \item $f(x)$ 是凸函数当且仅当对 $\forall\mathbf{x},\mathbf{y}\in C$, 有
              \begin{equation}
                  f(\mathbf{y})\geq f(\mathbf{x})+\nabla f(\mathbf{x})^{\prime}(\mathbf{y}-\mathbf{x})
              \end{equation}
        \item $f(\mathbf{x})$ 是严格凸函数当且仅当对 $\forall\mathbf{x},\mathbf{y}\in C,\mathbf{x}\neq\mathbf{y}$, 有
              \begin{equation}
                  f(\mathbf{y})>f(\mathbf{x})+\nabla f(\mathbf{x})^{\prime}(\mathbf{y}-\mathbf{x})
              \end{equation}
    \end{enumerate}
\end{theorem}

\begin{theorem}[凸函数的二阶判定条件]
    设集合 $C\subset\mathbb{R}^{n}$ 为非空开凸集,函数 $f:C\rightarrow\mathbb{R}$ 二阶连续可微,则
    \begin{enumerate}
        \item $f(x)$ 是凸函数当且仅当对 $\forall\mathbf{x}\in C$,Hessien 矩阵 $\boldsymbol{H}(\mathbf{x})$ 半正定;
        \item 若对 $\forall\mathbf{x}\in C$,Hessien 矩阵 $\boldsymbol{H}(\mathbf{x})$ 正定,则 $f$ 是严格凸函数。
    \end{enumerate}
\end{theorem}

\subsection{常用矩阵求导公式}

\begin{equation}
    \frac{\partial\boldsymbol{\beta}^{\prime}\mathbf{x}}{\partial\mathbf{x}}=\boldsymbol{\beta}
\end{equation}

\begin{equation}
    \frac{\partial\mathbf{x}^{\prime}\mathbf{x}}{\partial\mathbf{x}}=2\mathbf{x}
\end{equation}

\begin{equation}
    \frac{\partial\mathbf{x}^{\prime}\mathbf{A}\mathbf{x}}{\partial\mathbf{x}}=\left(\mathbf{A}+\mathbf{A}^{\prime}\right)\mathbf{x}
\end{equation}

\chapter{无约束最优化方法的基本结构}

\section{最优性条件}

\section{方法的特性}

对于无约束最优化问题
\begin{equation}
    \min_{\mathbf{x}\in\mathbb{R}^{n}}f\left(\mathbf{x}\right)
\end{equation}
一种常用的解决该问题的迭代算法如下:
\begin{quotation}
    选定初始点 $\mathbf{x}_{0}\in\mathbb{R}^{n}$,在 $\mathbf{x}_{0}$ 处,确定一个使函数值下降的方向 $\mathbf{d}$ 与步长 $\alpha$,继而求得下一个迭代点,以此类推,产生一个迭代点列 $\left\{\mathbf{x}_{k}\right\}$,$\left\{\mathbf{x}_{k}\right\}$ 或者其子列应收敛于最优解。当给定的某种终止准则满足时,或者 $\mathbf{x}_{k}$ 已满足要求的近似最优解的精度,或者算法已经无力进一步改善迭代点,则迭代结束。
\end{quotation}
\begin{remark}
    对于上述迭代算法,下降方向与步长的选取顺序不同,导致产生不同类型的方法。
    \begin{itemize}
        \item 线搜索方法:在 $\mathbf{x}_{k}$ 点求得下降方向 $\mathbf{d}_{k}$,再沿 $\mathbf{d}_{k}$ 确定步长 $\alpha_{k}$;
        \item 信赖域方法:在 $\mathbf{x}_{k}$ 点,先限定步长的范围,再同时确定下降方法 $\mathbf{d}_{k}$ 和步长 $\alpha_{k}$。
    \end{itemize}
\end{remark}

\section{线搜索准则}

\begin{example}
    考虑对正定二次函数 $f(\mathbf{x})=\frac{1}{2}\mathbf{x}^{\prime}\mathbf{G}\mathbf{x}+\mathbf{b}^{\prime} \mathbf{x}$,在点 $\mathbf{x}_{k}$,求出沿下降方向 $\mathbf{d}_{k}$ 作精确线搜索的步长 $\alpha_{k}$。
\end{example}

\begin{proof}
    若要给出沿下降方向 $\mathbf{d}_{k}$ 作精确线搜索的步长 $\alpha_{k}$,则应对于 $\mathbf{x}_{k+1}=\mathbf{x}_{k}+\alpha_{k}\mathbf{d}_{k}$,满足
    \begin{equation}
        \mathbf{d}_{k}^{\prime}\nabla f(\mathbf{x}_{k+1})=0
    \end{equation}

    由于 $f(\mathbf{x})$ 为正定二次函数,则 $\mathbf{G}$ 为对称正定矩阵,则有 $\mathbf{G}^{\prime}=\mathbf{G}$,因此
    \begin{equation}
        \nabla f(\mathbf{x})=\mathbf{G}\mathbf{x}+\mathbf{b}
    \end{equation}
    即
    \begin{equation}
        \mathbf{d}_{k}^{\prime}\nabla f(\mathbf{x}_{k+1})=\mathbf{d}_{k}^{\prime}\left[\mathbf{G}\left(\mathbf{x}_{k}+\alpha_{k}\mathbf{d}_{k}\right)+\mathbf{b}\right]=0
    \end{equation}

    由于 $\mathbf{G}$ 为正定矩阵,有 $\mathbf{d}_{k}^{\prime}\mathbf{G}\mathbf{d}_{k}>0$,因此,上式可化简为
    \begin{equation}
        \alpha_{k}=-\frac{\mathbf{d}_{k}^{\prime}\left(\mathbf{G}\mathbf{x}_{k}+\mathbf{b}\right)}{\mathbf{d}_{k}^{\prime}\mathbf{G}\mathbf{d}_{k}}=-\frac{\mathbf{d}_{k}^{\prime}\nabla f(\mathbf{x}_{k})}{\mathbf{d}_{k}^{\prime}\mathbf{G}\mathbf{d}_{k}}
    \end{equation}
\end{proof}

\section{线搜索求步长}

无约束最优化算法中线搜索方法的基本结构如下:

\begin{algorithm}
    \caption{线搜索方法的基本结构(P15)}
    \KwIn{目标函数 $f\left(\mathbf{x}\right)$,初始点 $\mathbf{x}_{0}\in\mathbb{R}^{n}$ 以及终止准则}
    \KwOut{最优解 $\mathbf{x}^{*}$ 以及 $f\left(\mathbf{x}^{*}\right)$}
    \For{$k=0,\ldots,$}{
        确定下降方向 $\mathbf{d}_{k}$,使得 $\nabla f\left(\mathbf{x}_{k}\right)^{\prime}\mathbf{d}_{k}<0$\;
        确定步长 $\alpha_{k}$ 使得 $f\left(\mathbf{x}_{k}+\alpha_{k}\mathbf{d}_{k}\right)<f\left(\mathbf{x}_{k}\right)$\;
        $\mathbf{x}_{k+1}\leftarrow\mathbf{x}_{k}+\alpha_{k}\mathbf{d}_{k}$\;
        \If{$\mathbf{x}_{k}$ 满足给定的终止准则}{
            break\;
        }
    }
    $\mathbf{x}^{*}\leftarrow\mathbf{x}_{k}$,$f\left(\mathbf{x}^{*}\right)\leftarrow f\left(\mathbf{x}_{k}\right)$\;
\end{algorithm}

\begin{listing}
    \begin{minted}{python}
def unconstrained_optimize(f, x0, epsilon=1e-8, max_iter=1000):
    x = np.empty((max_iter+1, x0.shape[0]))  # 定义 x 初始存储空间

    x[0] = x0
    for k in range(max_iter):
        d = search_desc_direction(f, x[k], ...)

        phi = lambda alpha: f(x[k] + alpha * d)
        alpha = search_step_length(phi, ...)

        x[k+1] = x[k] + alpha * d

        # if np.linalg.norm(g(x[k+1])) <= epsilon:
        # if f(x[k]) - f(x[k+1]) <= epsilon:
        if np.linalg.norm(x[k] - x[k+1]) <= epsilon:
            break

    return x[k+1], f(x[k+1])
    \end{minted}
    \caption{线搜索方法的基本结构:Python 实现}
\end{listing}

\subsection{精确线搜索方法}

\subsubsection{确定步长搜索区间}

从一点出发,按一定步长,试图确定函数值呈现“高-低-高”的三点,从而得到一个近似的单峰区间。

\begin{algorithm}
    \caption{进退法求初始搜索区间(P26)}
    \KwIn{一元函数 $\phi\left(\alpha\right)$,初始点 $\alpha_{0}$,初始步长 $\gamma_{0}$,步长因子 $t$}
    \KwOut{初始区间 $\left[a, b\right]$}
    \For{$i=0,\ldots,$}{
        $\alpha_{i+1}\leftarrow\alpha_{i}+\gamma_{i}$\;
        \eIf{$\phi\left(\alpha_{i+1}\right)\geq\phi\left(\alpha_{i}\right)$ 或 $\alpha_{i+1}\leq 0$}{
            \leIf{$i=0$}{$\gamma_{i+1}\leftarrow-\gamma_{i}$,$\alpha\leftarrow\alpha_{i+1}$}{break}
        }{
            $\gamma_{i+1}\leftarrow t\gamma_{i}$,$\alpha\leftarrow\alpha_{i}$,$\alpha_{i}\leftarrow\alpha_{i+1}$\;
        }
    }
    $a\leftarrow\min\left\{\alpha,\alpha_{i+1}\right\}$,$b\leftarrow\max\left\{\alpha,\alpha_{i+1}\right\}$\;
\end{algorithm}

\begin{remark}
    在该算法中 $\alpha_{i}$,$\alpha_{i+1}$, $\alpha$ 分别起到了什么作用?
\end{remark}

\begin{listing}
    \begin{minted}{python}
def search_unimodal_interval(phi, alpha0, gamma=0.1, t=2, max_iter=100):
    alphas = np.empty(max_iter + 1)

    alphas[0] = alpha0
    for i in range(max_iter):
        alphas[i+1] = alphas[i] + gamma
        if phi(alphas[i+1]) >= phi(alphas[i]) or alphas[i+1] <= 0:
            if i == 0:
                gamma = -gamma
                alpha = alphas[i+1]
            else:
                break
        else:
            gamma = t * gamma
            alpha = alphas[i]
            alphas[i] = alphas[i+1]

    return min(alpha, alphas[i+1]), max(alpha, alphas[i+1])
    \end{minted}
    \caption{进退法求初始搜索区间:Python 实现}
\end{listing}

\subsubsection{缩小步长搜索区间}

\paragraph{0.618 方法}

通过试探点函数值的比较,使包含极值点的搜索区间不断缩小。

\begin{algorithm}
    \caption{0.618 方法求一元函数 $\phi\left(\alpha\right)$ 的近似极小点(P27)}
    \KwIn{一元函数 $\phi\left(\alpha\right)$,初始搜索区间 $\left[a_{0},b_{0}\right]$ 满足 $a_{0}>b_{0}>0$,容许误差 $\varepsilon>0$}
    \KwOut{近似极小点 $\alpha^{*}$}
    $\tau\leftarrow \frac{\sqrt{5}-1}{2}\approx 0.618$\;
    \For{$i=0,\ldots,$}{
    $a_{i}^{l}\leftarrow a_{i}+\left(1-\tau\right)\left(b_{i}-a_{i}\right)$,$a_{i}^{r}\leftarrow a_{i}+\tau\left(b_{i}-a_{i}\right)$\;
    \eIf{$\phi\left(a_{i}^{l}\right)<\phi\left(a_{i}^{r}\right)$}{
        $a_{i+1}\leftarrow a_{i}$,$b_{i+1}\leftarrow a_{i}^{r}$\;
    }{
        $a_{i+1}\leftarrow a_{i}^{l}$,$b_{i+1}\leftarrow b_{i}$\;
    }
    \lIf{$b_{i+1}-a_{i+1}<\varepsilon$}{break}
    }
    $a^{*}\leftarrow\frac{b_{i}+a_{i}}{2}$\;
\end{algorithm}

\begin{listing}
    \begin{minted}{python}
def search_step_length_gold(phi, a0, b0, epsilon=1e-8, max_iter=1000):
    a, b = np.empty(max_iter + 1), np.empty(max_iter + 1)

    a[0], b[0] = a0, b0
    tau = (np.sqrt(5) - 1) / 2
    for i in range(max_iter):
        a_l = a[i] + (1 - tau) * (b[i] - a[i])
        a_r = a[i] + tau * (b[i] - a[i])
        if phi(a_l) < phi(a_r):
            a[i+1], b[i+1] = a[i], a_r
        else:
            a[i+1], b[i+1] = a_l, b[i]
        if b[i+1] - a[i+1] < epsilon:
            break
    return (a[i+1] + b[i+1]) / 2
    \end{minted}
    \caption{0.618 方法求一元函数 $\phi\left(\alpha\right)$ 的近似极小点:Python 实现}
\end{listing}

\paragraph{多项式插值法}

通过在搜索区间中不断地使用二次多项式去近似目标函数,并逐步用插值多项式的极小点去逼近线搜索问题的极小点。

设 $\alpha_{1}<\alpha_{2}<\alpha_{3}$,$\phi\left(\alpha_{1}\right)>\phi\left(\alpha_{2}\right)$,$\phi\left(\alpha_{2}\right)<\phi\left(\alpha_{3}\right)$, 拟合如下的二次插值多项式:
\begin{equation}
    p(\alpha)=a\alpha^{2}+b\alpha+c
\end{equation}
满足插值条件:
\begin{equation}
    \left\{\begin{array}{l}
        p\left(\alpha_{1}\right)=a\alpha_{1}^{2}+b\alpha_{1}+c=\phi\left(\alpha_{1}\right) \\
        p\left(\alpha_{2}\right)=a\alpha_{2}^{2}+b\alpha_{2}+c=\phi\left(\alpha_{2}\right) \\
        p\left(\alpha_{3}\right)=a\alpha_{3}^{2}+b\alpha_{3}+c=\phi\left(\alpha_{3}\right)
    \end{array}\right.
\end{equation}
从极值的必要条件知 $p^{\prime}(\alpha_{p})=2a\alpha_{p}+b=0$,求得
\begin{equation}
    \alpha_{p}=-\frac{b}{2a}
\end{equation}
从而可以算出
\begin{equation}
    \alpha_{p}=\frac{1}{2}\left(\alpha_{1}+\alpha_{2}-\frac{c_{1}}{c_{2}}\right)
\end{equation}
其中
\begin{equation}
    c_{1}=\frac{\phi\left(\alpha_{3}\right)-\phi\left(\alpha_{1}\right)}{\alpha_{3}-\alpha_{1}},\quad c_{2}=\frac{\frac{\phi\left(\alpha_{2}\right)-\phi\left(\alpha_{1}\right)}{\alpha_{2}-\alpha_{1}}-c_{1}}{\alpha_{2}-\alpha_{3}}
\end{equation}

\begin{algorithm}
    \caption{多项式插值法(三点二次插值法)求一维函数 $\phi\left(\alpha\right)$ 的近似极小点}
    \KwIn{一元函数 $\phi\left(\alpha\right)$,初始搜索区间 $\left[a_{0},b_{0}\right]$ 满足 $a_{0}>b_{0}>0$,容许误差 $\varepsilon>0$}
    \KwOut{近似极小点 $\alpha^{*}$}
    任取 $c_{0}\in\left[a_{0},b_{0}\right]$\;
    \For{$i=0,\ldots,$}{
    $\alpha_{p}\leftarrow\frac{1}{2}\left(a_{i}+b_{i}-\frac{c_{1}}{c_{2}}\right)$,其中 $c_{1}=\frac{\phi\left(c_{i}\right)-\phi\left(a_{i}\right)}{c_{i}-a_{i}},c_{2}=\frac{\frac{\phi\left(b_{i}\right)-\phi\left(a_{i}\right)}{b_{i}-a_{i}}-c_{1}}{b_{i}-c_{i}}$\;
    \eIf{$\phi\left(c_{i}\right)\leq\phi\left(\alpha_{p}\right)$}{
        \eIf{$c_{i}\leq\alpha_{p}$}{
            $a_{i+1}\leftarrow a_{i}$,$c_{i+1}\leftarrow c_{i}$,$b_{i+1}\leftarrow\alpha_{p}$\;
        }{
            $a_{i+1}\leftarrow\alpha_{p}$,$c_{i+1}\leftarrow c_{i}$,$b_{i+1}\leftarrow b_{i}$\;
        }
    }{
        \eIf{$c_{i}\leq\alpha_{p}$}{
            $a_{i+1}\leftarrow c_{i}$,$c_{i+1}\leftarrow\alpha_{p}$,$b_{i+1}\leftarrow b_{i}$\;
        }{
            $a_{i+1}\leftarrow a_{i}$,$c_{i+1}\leftarrow\alpha_{p}$,$b_{i+1}\leftarrow c_{i}$\;
        }
    }
    \lIf{$b_{i+1}-a_{i+1}<\varepsilon$}{break}
    }
    $\alpha^{*}\leftarrow\alpha_{p}$\;
\end{algorithm}
\begin{listing}
    \begin{minted}{python}
def search_step_length_poly32(phi, a0, b0, epsilon=1e-8, max_iter=1000):
    a, b, c = np.empty(max_iter + 1), np.empty(max_iter + 1), np.empty(max_iter + 1)

    a[0], c[0], b[0] = a0, (a0 + b0) / 2, b0
    for i in range(max_iter):
        c1 = (phi(c[i]) - phi(a[i])) / (c[i] - a[i])
        c2 = ((phi(b[i]) - phi(a[i])) / (b[i] - a[i]) - c1) / (b[i] - c[i])
        alpha_p = 0.5 * (a[i] + b[i] - c1 / c2)
        if phi(c[i]) <= phi(alpha_p):
            if c[i] <= alpha_p:
                a[i+1], c[i+1], b[i+1] = a[i], c[i], alpha_p
            else:
                a[i+1], c[i+1], b[i+1] = alpha_p, c[i], b[i]
        else:
            if c[i] <= alpha_p:
                a[i+1], c[i+1], b[i+1] = c[i], alpha_p, b[i]
            else:
                a[i+1], c[i+1], b[i+1] = a[i], alpha_p, c[i]
        if b[i+1] - a[i+1] < epsilon:
            break

    return alpha_p
    \end{minted}
    \caption{多项式插值法(三点二次插值法)求一维函数 $\phi\left(\alpha\right)$ 的近似极小点:Python 实现}
\end{listing}

\subsubsection{牛顿切线法}

\begin{algorithm}
    \caption{牛顿切线法}
    \KwIn{一元函数 $\phi\left(\alpha\right)$ 及其一阶导函数 $\phi^{\prime}\left(\alpha\right)$ 与二阶导函数 $\phi^{\prime\prime}\left(\alpha\right)$,初始点 $\alpha_{0}$,容许误差 $\varepsilon>0$}
    \KwOut{近似极小点 $\alpha^{*}$}
    \For{$i=0,\ldots,$}{
        $\alpha_{i+1}=\alpha_{i}-\frac{\phi^{\prime}\left(\alpha_{i}\right)}{\phi^{\prime\prime}\left(\alpha_{i}\right)}$\;
        \lIf{$|\alpha_{i+1}-\alpha_{i}|<\varepsilon$}{break}
    }
    $\alpha^{*}\leftarrow\alpha_{i+1}$\;
\end{algorithm}

\begin{listing}
    \begin{minted}{python}
def search_step_length_newton(phi, phi_grad, phi_hess, alpha0, epsilon=1e-8, max_iter=1000):
    alpha = np.empty(max_iter + 1)

    alpha[0] = alpha0
    for i in range(max_iter):
        alpha[i+1] = alpha[i] - phi_grad(alpha[i]) / phi_hess(alpha[i])
        if abs(alpha[i+1] - alpha[i]) < epsilon:
            break
    
    return alpha[i+1]
    \end{minted}
    \caption{牛顿切线法:Python 实现}
\end{listing}

\chapter{负梯度方法与 Newton 型方法}

\section{最速下降法}

\begin{example}
    用最速下降方法求解
    \begin{equation}
        \min\,f(\mathbf{x})=\frac{1}{2}\mathbf{x}^{\prime}\mathbf{G}\mathbf{x}+\mathbf{b}^{\prime}\mathbf{x}+c,\quad\mathbf{G}\ge
    \end{equation}
\end{example}

\begin{proof}
    对 $f(\mathbf{x})$ 求导可得,
    \begin{equation}
        g(\mathbf{x})=\mathbf{G}\mathbf{x}+\mathbf{b}
    \end{equation}
    因此,最速下降方向为
    \begin{equation}
        \mathbf{d}_{k}=-g(\mathbf{x}_{k})=-\left(\mathbf{G}\mathbf{x}_{k}+\mathbf{b}\right)
    \end{equation}

    通过一维精确线搜索求得步长 $\alpha_{k}$,即
    \begin{equation}
        \alpha_{k}=\arg\min_{\alpha}\,\phi(\alpha)
    \end{equation}
    其中,
    \begin{equation}
        \phi(\alpha)=f(\mathbf{x}_{k}+\alpha\mathbf{d}_{k})=\frac{1}{2}\left(\mathbf{x}_{k}+\alpha\mathbf{d}_{k}\right)^{\prime}\mathbf{G}\left(\mathbf{x}_{k}+\alpha\mathbf{d}_{k}\right)+\mathbf{b}^{\prime}\left(\mathbf{x}_{k}+\alpha\mathbf{d}_{k}\right)+c
    \end{equation}
    对 $\phi(\alpha_{k})$ 求导可得,
    \begin{equation}
        \begin{aligned}
            \phi^{\prime}(\alpha)= & \left[\mathbf{G}\left(\mathbf{x}_{k}+\alpha\mathbf{d}_{k}\right)\right]^{\prime}\mathbf{d}_{k}+\mathbf{b}^{\prime}\mathbf{d}_{k} \\
            =                      & \left[\mathbf{G}\left(\mathbf{x}_{k}+\alpha\mathbf{d}_{k}\right)+\mathbf{b}\right]^{\prime}\mathbf{d}_{k}                        \\
            =                      & \left[\left(\mathbf{G}\mathbf{x}_{k}+\mathbf{b}\right)+\alpha\mathbf{G}\mathbf{d}_{k}\right]^{\prime}\mathbf{d}_{k}              \\
            =                      & -\mathbf{d}_{k}^{\prime}\mathbf{d}_{k}+\alpha\mathbf{d}_{k}^{\prime}\mathbf{G}\mathbf{d}_{k}
        \end{aligned}
    \end{equation}
    令 $\phi^{\prime}(\alpha)=0$,可解得
    \begin{equation}
        \alpha_{k}=\frac{\mathbf{d}_{k}^{\prime}\mathbf{d}_{k}}{\mathbf{d}_{k}^{\prime}\mathbf{G}\mathbf{d}_{k}}
    \end{equation}

    因此,迭代公式为
    \begin{equation}
        \mathbf{x}_{k+1}=\mathbf{x}_{k}+\alpha_{k}\mathbf{d}_{k}=\mathbf{x}_{k}+\frac{\mathbf{d}_{k}^{\prime}\mathbf{d}_{k}}{\mathbf{d}_{k}^{\prime}\mathbf{G}\mathbf{d}_{k}}\mathbf{d}_{k}=\mathbf{x}_{k}-\frac{\mathbf{g}_{k}^{\prime}\mathbf{g}_{k}}{\mathbf{g}_{k}^{\prime}\mathbf{G}\mathbf{g}_{k}}\mathbf{g}_{k}
    \end{equation}
    其中,
    \begin{equation}
        \mathbf{g}_{k}=g(\mathbf{x_{k}})=\mathbf{G}\mathbf{x}_{k}+\mathbf{b}
    \end{equation}
\end{proof}

\end{document}