\documentclass[cn]{elegantpaper}
\usepackage[linesnumbered,lined,ruled]{algorithm2e}
\usepackage{listing,minted}
\setminted{
    frame=lines,
    framesep=2mm,
    baselinestretch=1.2,
    fontsize=\footnotesize,
    linenos
}
\title{无约束最优化方法的基本结构:编程实践}
\author{龚梓阳}
\date{\zhtoday}

\begin{document}

\maketitle

对于无约束最优化问题
\begin{equation}
    \min_{\mathbf{x}\in\mathbb{R}^{n}}f\left(\mathbf{x}\right)
\end{equation}
一种常用的解决该问题的迭代算法如下:
\begin{quotation}
    选定初始点 $\mathbf{x}_{0}\in\mathbb{R}^{n}$,在 $\mathbf{x}_{0}$ 处,确定一个使函数值下降的方向 $\mathbf{d}$ 与步长 $\alpha$,继而求得下一个迭代点,以此类推,产生一个迭代点列 $\left\{\mathbf{x}_{k}\right\}$,$\left\{\mathbf{x}_{k}\right\}$ 或者其子列应收敛于最优解。当给定的某种终止准则满足时,或者 $\mathbf{x}_{k}$ 已满足要求的近似最优解的精度,或者算法已经无力进一步改善迭代点,则迭代结束。
\end{quotation}
\begin{remark}
    对于上述迭代算法,下降方向与步长的选取顺序不同,导致产生不同类型的方法。
    \begin{itemize}
        \item 线搜索方法:在 $\mathbf{x}_{k}$ 点求得下降方向 $\mathbf{d}_{k}$,再沿 $\mathbf{d}_{k}$ 确定步长 $\alpha_{k}$;
        \item 信赖域方法:在 $\mathbf{x}_{k}$ 点,先限定步长的范围,再同时确定下降方法 $\mathbf{d}_{k}$ 和步长 $\alpha_{k}$。
    \end{itemize}
\end{remark}

\section{线搜索方法}

无约束最优化算法中线搜索方法的基本结构如下:

\begin{algorithm}[H]
    \caption{线搜索方法的基本结构(P15)}
    \KwIn{目标函数 $f\left(\mathbf{x}\right)$,初始点 $\mathbf{x}_{0}\in\mathbb{R}^{n}$ 以及终止准则}
    \KwOut{最优解 $\mathbf{x}^{*}$ 以及 $f\left(\mathbf{x}^{*}\right)$}
    \For{$k=0,\ldots,$}{
        确定下降方向 $\mathbf{d}_{k}$,使得 $\nabla f\left(\mathbf{x}_{k}\right)^{\prime}\mathbf{d}_{k}<0$\;
        确定步长 $\alpha_{k}$ 使得 $f\left(\mathbf{x}_{k}+\alpha_{k}\mathbf{d}_{k}\right)<f\left(\mathbf{x}_{k}\right)$\;
        $\mathbf{x}_{k+1}\leftarrow\mathbf{x}_{k}+\alpha_{k}\mathbf{d}_{k}$\;
        \If{$\mathbf{x}_{k}$ 满足给定的终止准则}{
            break\;
        }
    }
    $\mathbf{x}^{*}\leftarrow\mathbf{x}_{k}$,$f\left(\mathbf{x}^{*}\right)\leftarrow f\left(\mathbf{x}_{k}\right)$\;
\end{algorithm}

\begin{remark}
    \texttt{while} 与 \texttt{for} 有什么区别?
\end{remark}

\begin{listing}[H]
    \begin{minted}{python}
def unconstrained_optimize(f, x0, epsilon=1e-8, max_iter=1000):
    x = np.empty((max_iter+1, x0.shape[0]))  # 定义 x 初始存储空间

    x[0] = x0
    for k in range(max_iter):
        d = search_desc_direction(f, x[k], ...)

        phi = lambda alpha: f(x[k] + alpha * d)
        alpha = search_step_length(phi, ...)

        x[k+1] = x[k] + alpha * d

        # if np.linalg.norm(g(x[k+1])) <= epsilon:
        # if f(x[k]) - f(x[k+1]) <= epsilon:
        if np.linalg.norm(x[k] - x[k+1]) <= epsilon:
            break

    return x[k+1], f(x[k+1])
    \end{minted}
    \caption{线搜索方法的基本结构:Python 实现}
\end{listing}

\subsection{精确线搜索方法}

% 为了更好地说明算法,我们考虑书中一道习题:
% \begin{equation}
%     \phi(\alpha)=1-\alpha\mathrm{exp}\left(-\alpha^{2}\right)
% \end{equation}

% \begin{figure}[htbp]
%     \centering
%     \includegraphics[width=.7\linewidth]{figures/exercise-2.8.png}
%     \caption{习题 2.8}
% \end{figure}

\subsubsection{确定步长搜索区间}

从一点出发,按一定步长,试图确定函数值呈现“高-低-高”的三点,从而得到一个近似的单峰区间。

\begin{algorithm}[H]
    \caption{进退法求初始搜索区间(P26)}
    \KwIn{一元函数 $\phi\left(\alpha\right)$,初始点 $\alpha_{0}$,初始步长 $\gamma_{0}$,步长因子 $t$}
    \KwOut{初始区间 $\left[a, b\right]$}
    \For{$i=0,\ldots,$}{
        $\alpha_{i+1}\leftarrow\alpha_{i}+\gamma_{i}$\;
        \eIf{$\phi\left(\alpha_{i+1}\right)\geq\phi\left(\alpha_{i}\right)$ 或 $\alpha_{i+1}\leq 0$}{
            \leIf{$i=0$}{$\gamma_{i+1}\leftarrow-\gamma_{i}$,$\alpha\leftarrow\alpha_{i+1}$}{break}
        }{
            $\gamma_{i+1}\leftarrow t\gamma_{i}$,$\alpha\leftarrow\alpha_{i}$,$\alpha_{i}\leftarrow\alpha_{i+1}$\;
        }
    }
    $a\leftarrow\min\left\{\alpha,\alpha_{i+1}\right\}$,$b\leftarrow\max\left\{\alpha,\alpha_{i+1}\right\}$\;
\end{algorithm}

\begin{remark}
    在该算法中 $\alpha_{i}$,$\alpha_{i+1}$, $\alpha$ 分别起到了什么作用?
\end{remark}

\begin{listing}[H]
    \begin{minted}{python}
def search_unimodal_interval(phi, alpha0, gamma=0.1, t=2, max_iter=100):
    alphas = np.empty(max_iter + 1)

    alphas[0] = alpha0
    for i in range(max_iter):
        alphas[i+1] = alphas[i] + gamma
        if phi(alphas[i+1]) >= phi(alphas[i]) or alphas[i+1] <= 0:
            if i == 0:
                gamma = -gamma
                alpha = alphas[i+1]
            else:
                break
        else:
            gamma = t * gamma
            alpha = alphas[i]
            alphas[i] = alphas[i+1]

    return min(alpha, alphas[i+1]), max(alpha, alphas[i+1])
    \end{minted}
    \caption{进退法求初始搜索区间:Python 实现}
\end{listing}

\begin{remark}
    在该代码中,是否有必要存储每一个 $\alpha_{i}$ 么?
\end{remark}

\subsubsection{缩小步长搜索区间}

\paragraph{0.618 方法}

通过试探点函数值的比较,使包含极值点的搜索区间不断缩小。

\begin{algorithm}[H]
    \caption{0.618 方法求一元函数 $\phi\left(\alpha\right)$ 的近似极小点(P27)}
    \KwIn{一元函数 $\phi\left(\alpha\right)$,初始搜索区间 $\left[a_{0},b_{0}\right]$ 满足 $a_{0}>b_{0}>0$,容许误差 $\varepsilon>0$}
    \KwOut{近似极小点 $\alpha^{*}$}
    $\tau\leftarrow \frac{\sqrt{5}-1}{2}\approx 0.618$\;
    \For{$i=0,\ldots,$}{
    $a_{i}^{l}\leftarrow a_{i}+\left(1-\tau\right)\left(b_{i}-a_{i}\right)$,$a_{i}^{r}\leftarrow a_{i}+\tau\left(b_{i}-a_{i}\right)$\;
    \eIf{$\phi\left(a_{i}^{l}\right)<\phi\left(a_{i}^{r}\right)$}{
        $a_{i+1}\leftarrow a_{i}$,$b_{i+1}\leftarrow a_{i}^{r}$\;
    }{
        $a_{i+1}\leftarrow a_{i}^{l}$,$b_{i+1}\leftarrow b_{i}$\;
    }
    \lIf{$b_{i+1}-a_{i+1}<\varepsilon$}{break}
    }
    $a^{*}\leftarrow\frac{b_{i}+a_{i}}{2}$\;
\end{algorithm}

\begin{listing}[H]
    \begin{minted}{python}
def search_step_length_gold(phi, a0, b0, epsilon=1e-8, max_iter=1000):
    a, b = np.empty(max_iter + 1), np.empty(max_iter + 1)

    a[0], b[0] = a0, b0
    tau = (np.sqrt(5) - 1) / 2
    for i in range(max_iter):
        a_l = a[i] + (1 - tau) * (b[i] - a[i])
        a_r = a[i] + tau * (b[i] - a[i])
        if phi(a_l) < phi(a_r):
            a[i+1], b[i+1] = a[i], a_r
        else:
            a[i+1], b[i+1] = a_l, b[i]
        if b[i+1] - a[i+1] < epsilon:
            break
    return (a[i+1] + b[i+1]) / 2
    \end{minted}
    \caption{0.618 方法求一元函数 $\phi\left(\alpha\right)$ 的近似极小点:Python 实现}
\end{listing}

\begin{remark}
    在该代码中,是否利用了0.618 法简化计算的目的(每次少算一次 $\phi\left(\cdot\right)$)?如果没有,我们应该怎么解决呢?
\end{remark}

\paragraph{多项式插值法}

通过在搜索区间中不断地使用二次多项式去近似目标函数,并逐步用插值多项式的极小点去逼近线搜索问题的极小点。

设 $\alpha_{1}<\alpha_{2}<\alpha_{3}$,$\phi\left(\alpha_{1}\right)>\phi\left(\alpha_{2}\right)$,$\phi\left(\alpha_{2}\right)<\phi\left(\alpha_{3}\right)$, 拟合如下的二次插值多项式:
\begin{equation}
    p(\alpha)=a\alpha^{2}+b\alpha+c
\end{equation}
满足插值条件:
\begin{equation}
    \left\{\begin{array}{l}
        p\left(\alpha_{1}\right)=a\alpha_{1}^{2}+b\alpha_{1}+c=\phi\left(\alpha_{1}\right) \\
        p\left(\alpha_{2}\right)=a\alpha_{2}^{2}+b\alpha_{2}+c=\phi\left(\alpha_{2}\right) \\
        p\left(\alpha_{3}\right)=a\alpha_{3}^{2}+b\alpha_{3}+c=\phi\left(\alpha_{3}\right)
    \end{array}\right.
\end{equation}
从极值的必要条件知 $p^{\prime}(\alpha_{p})=2a\alpha_{p}+b=0$,求得
\begin{equation}
    \alpha_{p}=-\frac{b}{2a}
\end{equation}
从而可以算出
\begin{equation}
    \alpha_{p}=\frac{1}{2}\left(\alpha_{1}+\alpha_{2}-\frac{c_{1}}{c_{2}}\right)
\end{equation}
其中
\begin{equation}
    c_{1}=\frac{\phi\left(\alpha_{3}\right)-\phi\left(\alpha_{1}\right)}{\alpha_{3}-\alpha_{1}},\quad c_{2}=\frac{\frac{\phi\left(\alpha_{2}\right)-\phi\left(\alpha_{1}\right)}{\alpha_{2}-\alpha_{1}}-c_{1}}{\alpha_{2}-\alpha_{3}}
\end{equation}

\begin{algorithm}[H]
    \caption{多项式插值法(三点二次插值法)求一维函数 $\phi\left(\alpha\right)$ 的近似极小点}
    \KwIn{一元函数 $\phi\left(\alpha\right)$,初始搜索区间 $\left[a_{0},b_{0}\right]$ 满足 $a_{0}>b_{0}>0$,容许误差 $\varepsilon>0$}
    \KwOut{近似极小点 $\alpha^{*}$}
    任取 $c_{0}\in\left[a_{0},b_{0}\right]$\;
    \For{$i=0,\ldots,$}{
    $\alpha_{p}\leftarrow\frac{1}{2}\left(a_{i}+b_{i}-\frac{c_{1}}{c_{2}}\right)$,其中 $c_{1}=\frac{\phi\left(c_{i}\right)-\phi\left(a_{i}\right)}{c_{i}-a_{i}},c_{2}=\frac{\frac{\phi\left(b_{i}\right)-\phi\left(a_{i}\right)}{b_{i}-a_{i}}-c_{1}}{b_{i}-c_{i}}$\;
    \eIf{$\phi\left(c_{i}\right)\leq\phi\left(\alpha_{p}\right)$}{
        \eIf{$c_{i}\leq\alpha_{p}$}{
            $a_{i+1}\leftarrow a_{i}$,$c_{i+1}\leftarrow c_{i}$,$b_{i+1}\leftarrow\alpha_{p}$\;
        }{
            $a_{i+1}\leftarrow\alpha_{p}$,$c_{i+1}\leftarrow c_{i}$,$b_{i+1}\leftarrow b_{i}$\;
        }
    }{
        \eIf{$c_{i}\leq\alpha_{p}$}{
            $a_{i+1}\leftarrow c_{i}$,$c_{i+1}\leftarrow\alpha_{p}$,$b_{i+1}\leftarrow b_{i}$\;
        }{
            $a_{i+1}\leftarrow a_{i}$,$c_{i+1}\leftarrow\alpha_{p}$,$b_{i+1}\leftarrow c_{i}$\;
        }
    }
    \lIf{$b_{i+1}-a_{i+1}<\varepsilon$}{break}
    }
    $\alpha^{*}\leftarrow\alpha_{p}$\;
\end{algorithm}
\begin{listing}[H]
    \begin{minted}{python}
def search_step_length_poly32(phi, a0, b0, epsilon=1e-8, max_iter=1000):
    a, b, c = np.empty(max_iter + 1), np.empty(max_iter + 1), np.empty(max_iter + 1)

    a[0], c[0], b[0] = a0, (a0 + b0) / 2, b0
    for i in range(max_iter):
        c1 = (phi(c[i]) - phi(a[i])) / (c[i] - a[i])
        c2 = ((phi(b[i]) - phi(a[i])) / (b[i] - a[i]) - c1) / (b[i] - c[i])
        alpha_p = 0.5 * (a[i] + b[i] - c1 / c2)
        if phi(c[i]) <= phi(alpha_p):
            if c[i] <= alpha_p:
                a[i+1], c[i+1], b[i+1] = a[i], c[i], alpha_p
            else:
                a[i+1], c[i+1], b[i+1] = alpha_p, c[i], b[i]
        else:
            if c[i] <= alpha_p:
                a[i+1], c[i+1], b[i+1] = c[i], alpha_p, b[i]
            else:
                a[i+1], c[i+1], b[i+1] = a[i], alpha_p, c[i]
        if b[i+1] - a[i+1] < epsilon:
            break

    return alpha_p
    \end{minted}
    \caption{多项式插值法(三点二次插值法)求一维函数 $\phi\left(\alpha\right)$ 的近似极小点:Python 实现}
\end{listing}

\subsubsection{牛顿切线法}

\begin{algorithm}[H]
    \caption{牛顿切线法}
    \KwIn{一元函数 $\phi\left(\alpha\right)$ 及其一阶导函数 $\phi^{\prime}\left(\alpha\right)$ 与二阶导函数 $\phi^{\prime\prime}\left(\alpha\right)$,初始点 $\alpha_{0}$,容许误差 $\varepsilon>0$}
    \KwOut{近似极小点 $\alpha^{*}$}
    \For{$i=0,\ldots,$}{
        $\alpha_{i+1}=\alpha_{i}-\frac{\phi^{\prime}\left(\alpha_{i}\right)}{\phi^{\prime\prime}\left(\alpha_{i}\right)}$\;
        \lIf{$|\alpha_{i+1}-\alpha_{i}|<\varepsilon$}{break}
    }
    $\alpha^{*}\leftarrow\alpha_{i+1}$\;
\end{algorithm}

\begin{listing}[H]
    \begin{minted}{python}
def search_step_length_newton(phi, phi_grad, phi_hess, alpha0, epsilon=1e-8, max_iter=1000):
    alpha = np.empty(max_iter + 1)

    alpha[0] = alpha0
    for i in range(max_iter):
        alpha[i+1] = alpha[i] - phi_grad(alpha[i]) / phi_hess(alpha[i])
        if abs(alpha[i+1] - alpha[i]) < epsilon:
            break
    
    return alpha[i+1]
    \end{minted}
    \caption{牛顿切线法:Python 实现}
\end{listing}

\end{document}