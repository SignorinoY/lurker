\documentclass[cn,12pt,mtpro2]{elegantpaper}
\usepackage{pgfplots}
\title{优化方法\,期末论文题目}
\author{}
\date{\zhtoday}

\begin{document}
\maketitle

\noindent 本次作业请于 \textbf{2022年1月9日前} 提交,需包含以下内容:
\begin{itemize}
    \item 纸质版论文:需包括摘要、引言、文献综述、正文、结论以及参考文献等部分。
    \item 补充材料:包括电子版论文(\texttt{.pdf} 格式)及相关补充材料(补充数据及\textbf{带有注释说明}的源代码等)。请将补充材料打包并命名为 “\textbf{期末论文-题目1/题目2-学号1-姓名1-学号2-姓名2-……}” 后发送至助教邮箱:\href{mailto:meetziyang@outlook.com}{meetziyang@outlook.com}。
\end{itemize}

\noindent 可以以小组形式进行(最多 5 人);两道题目任选其一即可;可使用任意编程语言,如无特殊要求,可使用现成软件包。


\section{拟合广义可加模型}

广义可加模型具有如下形式:
\begin{equation}
    f(\mathbf{x})=\alpha+\sum_{j=1}^{p}f_{j}(x_{j})
\end{equation}
其中,$\mathbf{x}\in\mathbb{R}^{p}$ 是自变量;$\alpha\in\mathbb{R}$ 是截距项,$f_{j}:\mathbb{R}\rightarrow\mathbb{R}$ 是未定义的(非参)光滑函数且 $f_{j}(0)=0$。

为简化问题,我们限制 $f_{j}(\cdot),j=1,\ldots,p$ 为分段线性函数,并给定结点 $-\infty<p_{1}<\ldots<p_{K}<\infty$,也就是说函数 $f_{j}(\cdot)$ 在区间 $(-\infty,p_{1}],[p_{1},p_{2}],\ldots,[p_{K},\infty)$ 上为线性函数并且在结点处连续(如图 \ref{figure:piecewise-affine-function} 所示)。对于该广义可加模型 $f(\mathbf{x})$,定义 $C$ 为所有分段线性函数 $f_{j}(\cdot)$ 在所有结点附近斜率的变化程度。该指标 $C$ 度量所有分段线性函数 $f_{j}(\cdot)$ 的非线性程度,当 $C=0$ 时,该广义可加模型退化为线性回归模型。

\begin{figure}[htp]
    \centering
    \begin{tikzpicture}[
            declare function={
                    func(\x)= (\x<=4) * (10+5*\x)   +
                    and (\x>=4, \x<7) * (30-2*(\x-4))     +
                    and (\x>=7,  \x<=10) * (24-10*(\x-7)) ;
                }
        ]    \begin{axis}[ticks=none,
                axis x line=middle, axis y line=middle,
                ymin=0, ymax=35, ytick={}, ylabel=$f_{j}(\cdot)$,
                xmin=0, xmax=10, xtick={}, xlabel=$\mu$,
            ]
            \addplot[blue, samples at ={0,1,2,3,3.99,4.01,6.99,7.01,10}]{func(x)};
            \filldraw (axis cs: 4,0) circle (1pt) node [above,font=\tiny] {$p_{1}$};
            \filldraw (axis cs: 7,0) circle (1pt) node [above,font=\tiny] {$p_{2}$};
        \end{axis}
    \end{tikzpicture}
    \caption{分段线性函数 $f_{j}(\cdot)$ 示意图}
    \label{figure:piecewise-affine-function}
\end{figure}
现给定一组数据 $(\mathbf{x}^{(1)},y^{(1)}),\ldots,(\mathbf{x}^{(n)},y^{(n)})\in\mathbb{R}^{p}\times\mathbb{R}$,我们希望该广义可加模型能够尽可能地拟合给定数据,即
\begin{equation}
    \min_{\alpha,f_{j}}\quad\sum_{i=1}^{n}\left[y^{(i)}-\alpha-\sum_{j=1}^{p}f_{j}(\mathbf{x}^{(i)}_{j})\right]^{2}+\lambda C
\end{equation}
其中,$\lambda>0$ 为正则化参数。

\begin{enumerate}
    \item 考虑如何表达该分段线性函数 $f_{j}(\cdot)$ 并定义 $f_{j}(\cdot)$ 的非线性程度,从而可以使用本学期内学过的优化方法求解该问题。
    \item 基于第(1)问中提出的方法,选择正则化参数 $\lambda$ 对 \texttt{problem1.py} 中生成的数据进行拟合。该文件包括一个 $n\times p$ 矩阵 $\mathbf{X}$(矩阵中每一行是 $(\mathbf{x}^{i})^{\mathrm{T}}$),一个列向量 $\mathbf{y}$ (向量中每一行是 $y^{(i)}$)以及一个向量 $\mathbf{p}$ 为给定的结点。给出求解得出的均方误差,并对比估计得到的函数 $\hat{f}_{j}(\cdot)$ 与真实的函数 $f_{j}(\cdot)$。(真实函数以及绘图程序在文件中已给出)
    \item 基于实现的方法(需自己确定分段线性函数结点 $\mathbf{p}$ 及正则化参数 $\lambda$),选择任意一真实数据\footnote{参考 \url{https://archive.ics.uci.edu/ml/index.php}。}进行拟合。
    \item 查阅相关资料,考虑其他类型的光滑函数(如样条函数等),重新求解第(1)---(3)问,并与给定的分段线性函数进行对比和讨论。
\end{enumerate}

\begin{note}
    分段线性函数 $f_{j}(\cdot)$ 可以表示为一系列基函数的线性组合
    \begin{equation}
        f_{j}(\mu)=\sum_{k=0}^{K}\beta_{jk}b_{k}(\mu)
    \end{equation}
    其中,
    \begin{equation}
        b_{0}(\mu)=\mu,\quad b_{k}(\mu)=(\mu-p_{k})_{+}-(-p_{k})_{+},\,k=1,\ldots,K,\quad(\mu)_{+}=\max(\mu,0)
    \end{equation}
\end{note}

\section{投资组合优化问题}

现有上证 50 的 50 支股票在过去一年内的价格记录,利用该数据可计算出第 $i$ 只股票的历史平均收益 $\mu_{i}$ 和上证 50 的相关系数矩阵 $\mathbf{C}$。我们可以依靠这些信息构建一个投资组合,我们用 $x_{i}$ 代表第 $i$ 支股票在投资组合中所占的比例,令 $\mathbf{x}=
    \left(x_{1},\ldots,x_{50}\right)^{\mathrm{T}}$,其中 $\sum_{i=1}^{50}x_{i}=1$。我们可以的计算得到投资组合的平均收益和方差分别为
\begin{equation}
    \mathbf{\mu}_{x}=\mathbf{\mu}^{\mathrm{T}}\mathbf{x},\quad\mathbf{V}_{\mathbf{x}}=\mathbf{x}^{\mathrm{T}}\mathbf{C}\mathbf{x}
\end{equation}
则我们可以构建一个最大化收益,同时最小化方差的投资组合:
\begin{equation}
    \begin{aligned}
        \min_{\mathbf{x}}\quad & q(\mathbf{x})=-\mathbf{\mu}^{\mathrm{T}}\mathbf{x}+\lambda\mathbf{x}^{\mathrm{T}}\mathbf{C}\mathbf{x} \\
        \textrm{s.t.}\quad     & \sum_{i=1}^{50}x_i=1                                                                                  \\
                               & x_{i}\geq 0,\quad i=1,\ldots,50
    \end{aligned}
\end{equation}
\begin{enumerate}
    \item 写出优化问题的拉格朗日乘子形式与 KKT 条件,并解释 KKT 条件的含义。
    \item 使用已有的优化算法包,编写代码求解上述问题,并比较不同 $\lambda$ 下的最优解 $q^{*}$ 的变化。
    \item 尝试自己编写优化算法求解问题,并验证结果。
\end{enumerate}
\end{document}