% !TEX program = xelatex

\documentclass[cn,a4paper,12pt,founder,mtpro2]{elegantpaper}
\title{负梯度方法与 Newton 型方法:上机作业}
\author{龚梓阳}
\date{\zhtoday}

\begin{document}

\maketitle

\begin{quotation}
    本次作业两道题目任选其一即可,可使用任意编程语言(如,Python,MATLAB,R等)。请各位同学于 \textbf{12月3日前} 提交以下内容:
    \begin{itemize}
        \item 纸质版报告:包括对优化问题的推导、题目所要求的讨论结果(以表格或图片形式展示并辅以一段文字说明)以及算法实现中遇到的问题与解决方案等。
        \item 代码:将\textbf{带有注释说明的源代码}打包并命名为“\textbf{学号-姓名-上机作业1}”后发送至助教邮箱:\href{mailto:meetziyang@outlook.com}{meetziyang@outlook.com}。
    \end{itemize}
\end{quotation}

\begin{enumerate}
    \item 负梯度方法与 Newton 型方法的数值比较。编写下列程序(至少需实现 \textbf{5 种方法},且必须包含\textbf{加粗}方法):
          \begin{itemize}
              \item 负梯度方法:\textbf{最速下降方法}、最小梯度方法和 BB 方法(BB1\&BB2)。
              \item Newton 方法:\textbf{基本牛顿方法}、阻尼 Newton 方法、混合方法和 LM 方法。
              \item 拟 Newton 方法:SR1 方法、DFP 方法和\textbf{BFGS 方法}。
          \end{itemize}

          考虑最优化问题
          \begin{equation}
              \min_{\mathbf{x}}\,f(\mathbf{x})=\sum_{i=1}^{n/2}\left[100\left(x_{2i-1}^{2}-x_{2i}\right)^{2}+\left(x_{2i-1}-1\right)^{2}\right]
          \end{equation}
          选择不同的问题规模($n=2,10,50$),对不同方法的有效性进行讨论。其中,
          \begin{itemize}
              \item 迭代的终止准则为 $f_{k-1}-f_{k}\leq 10^{-8}$,迭代次数上限为 1000。
              \item 讨论内容可以包括收敛结果,算法的迭代次数、函数调用次数、导数调用次数及 CPU 时间等。
          \end{itemize}

    \item 由人工神经网络方法解微分方程导出的最优化问题\footnote{第 2 题可使用现成软件包,如 Python(SciPy)、MATLAB(Optimization Toolbox)等。}。考虑以下常微分方程问题
          \begin{equation}
              \frac{\mathrm{d}y(x)}{\mathrm{d}x}=x-y+1,\quad y(0)=1
          \end{equation}
          其精确解为 $y(x)=x+\mathrm{e}^{-x}$。使用 1-n-1 前馈神经网络 $N(x;\mathbf{p})$ 在 $x\in[0,1]$ 范围内,求解其实验解,即
          \begin{equation}
              y_{t}(x;\mathbf{p})=y_0+xN(x;\mathbf{p}),\quad x\in[0,1]
          \end{equation}
          该问题最终可转换为如下优化问题
          \begin{equation}
              \min_{\mathbf{p}}\sum_{i=1}^{m}\left\{-x_{i}+\left[\sum_{j=1}^{n}\left(1+x_{i}\right)\frac{v_{j}}{1+\mathrm{e}^{\theta_{j}-w_{j}x_{i}}}+x_{i}\frac{v_{j}w_{j}\mathrm{e}^{\theta_{j}-w_{j}x_{i}}}{\left(1+\mathrm{e}^{\theta_{j}-w_{j}x_{i}}\right)^{2}}\right]\right\}^{2}
          \end{equation}
          其中,$\mathbf{p}=\left(\mathbf{w},\mathbf{v},\boldsymbol{\theta}\right)^{\prime}$ 为该 1-n-1 前馈神经网络的参数,$x_{1},x_{2},\ldots,x_{m}\in[0,1]$,为方便起见,令 $x_{i}$ 取 $[0,1]$ 上均匀分布的点。使用任意方法求解该优化问题,并讨论在不同网络节点数 ($n=2,3,4$)和训练元素个数($m=3,5,10$)的情况下,实验解 $y_{t}(x;\mathbf{p})$ 与精确解 $y(x)$ 的差异。
          \begin{note}
              考虑一阶常微分方程
              \begin{equation}
                  \frac{\mathrm{d}y(x)}{\mathrm{d}x}=f(x,y),\quad x\in[0,1]
                  \label{eq:1ode}
              \end{equation}

              假设其实验解 $y_{t}(x)$ 具有如下形式
              \begin{equation}
                  y_{t}(x;\mathbf{p})=y_0+xN(x;\mathbf{p})
              \end{equation}

              $N(x;\mathbf{p})$ 为 1-n-1 前馈神经网络,具有如下形式:
              \begin{equation}
                  N(x;\mathbf{p})=\sum_{j=1}^{n}v_{j}\varphi(z_{j}),\quad z_{j}=w_{j}x-\theta_{j}
              \end{equation}
              其中,$\mathbf{p}=\left(\mathbf{w},\mathbf{v},\boldsymbol{\theta}\right)^{\prime}$,$w_{j}$ 是从输入节点 $x$ 到隐藏层第 $j$ 个节点的权重,$v_{j}$ 是隐藏层第 $j$ 个节点到输出节点的权重,$\theta_{j}$ 是隐藏层第 $j$ 个节点的阈值,$\varphi(z)=\frac{1}{1+\mathrm{e}^{-z}}$ 为 Sigmoid 型激活函数。

              我们希望对于实验解 $y_{t}(x)$,它能够尽可能地满足一阶常微分方程(\ref{eq:1ode}),即
              \begin{equation}
                  \left[\frac{\mathrm{d}y_{t}(x;\mathbf{p})}{\mathrm{d}x}-f\left(x,y_{t}(x;\mathbf{p})\right)\right]^{2}\rightarrow 0,\quad x\in[0,1]
              \end{equation}
              
              因此,将 $x$ 在区间 $[0,1]$ 离散化,取 $0=x_{1},x_{2},\ldots,x_{m}=1$,得到如下优化目标函数:
              \begin{equation}
                  \min_{\mathbf{p}}\,\sum_{i=1}^{m}\left[\left.\frac{\mathrm{d}y_{t}(x;\mathbf{p})}{\mathrm{d}x}\right|_{x=x_{i}}-f\left(x_{i},y_{t}(x_{i};\mathbf{p})\right)\right]^{2}
              \end{equation}
          \end{note}
\end{enumerate}

\end{document}